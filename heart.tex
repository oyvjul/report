\section{introduction to the heart and its functions}
%%\label{sec:multicore_architectures}
%%Maybe quotes instead?
The heart. The most iconic organ in the human body, responsible for thousands of bad love songs and poorly made decisions. While in reality, the heart is actually a big muscular pump that powers the entire circulatory system, transporting mainly oxygen, nutrients, hormones and heat throughout the body. 

\subsection{Layers of the Heart Wall}
The heart consists of three layers, the \textit{epicardium}, \textit{myocardium}, and the \textit{endocardium}. 

The outermost layer is called the epicardium. The epicardium is a thin layer of connective tissue [19], and fat that serves as a protection from trauma as well as lubricant for the sorrounding environment. 

The heart is constructed with a special kind of muscle tissue called the myocardium, also referred to as the cardiac muscle. Which makes up the majority of the thickness and mass of the heart. The cardiac muscle is one of the main reasons to why the heart pumps blood. When the cardiac muscle cells receives electrictrial stimulation, it results in a rhytmic contraction and relaxtion phase, that makes the heart beat. A deeper description of the electrophysiological process in the heart can be found in section [X.X]

The endocardium is the innermost layer in the heart wall. bla bla

%%which is one of four tissues that supports, connects or separates different types of organs and tissues in the body [19] FOOTNOTE?

\subsection{Composition of the Heart}
To get an illustrative perspective of the composition of the heart, it helps think of it as a student apartment that is made up of a left and right unit. The left and right unit are seperated by a wall known as a cardiac septum. Student housings are known to be space saving, so each unit is then subdivided into a upper chamber referred to as the \textit{atrium}, and a lower chamber known as the \textit{ventricle}, as illustrated in [FIGURE X.X]. The right atrium (RA) is above the right ventricle (RV), while the left atrium (LA) sits above the left ventricle (LV), as illustrated in figur [X.X]. 

If we were to take our student apartment example even further, the only way of getting into or leaving the lower chambers is by passing the valves. It helps think of the valves as one-way entranced doors that only has two states, open and closed. Since the valves only opens and closes in one specific direction, they therefore prevents blood from flowing backwards. The valves are made of connective tissue [https://www.cardiosmart.org/heart-basics/how-the-heart-works][FOOTNOTE] that acts like flaps. The valves can be divided into two types, \textit{semilunar} and \textit{atrioventricular} valves [http://www.biosbcc.net/b100cardio/htm/heartant.htm]. The atrioventricular valve (AV) separates the atria from the ventricles. The  AV that separates the RA and RV is called the triscupid valve, due to its three cusps

The atria is responsible for receiving blood, meaning that the RA and LA are connected to veins which carry blood to the heart. The RA receives blood from the vein called the vena cava, while the LA receives blood from the four pulmonary veins which stems from the lungs. The ventricles are responsible for pushing blood away from the heart. The RV pushes blood through the pulmonary artery, and into the lungs, while the LV is responsible for pushing the blood into the aorta, and throughout the entire body.

The right side of the heart consists of a thinner myocardium compared to the left side. The difference in size is related to its functions. The right side of the heart ensures a pulmonary circulation by sending blood to the lungs. The left side of the heart sends blood through the aortic vessel and all the way to the extremeties of the body. Since blood that originates from the left side travels a greater distance compared to the right side, it needs to be sent away with greater force, and thus have a thicker myocardium.

%%If we were to take our student apartment example even further, the only entrance for getting into the left unit is located in the right atrium. This is a one-way entrance, meaning that students can only enter 
%%[http://www.hopkinsmedicine.org/healthlibrary/conditions/cardiovascular_diseases/anatomy_and_function_of_the_heart_valves_90,P03059/]
%%bilde her!

\begin{figure}[h]
 \centering 
     \includegraphics[width=0.9\textwidth]{bilder/b_heart_structure_new}
     \caption{Explaining the decomposition process in a 3 dimensional structured mesh. Illustration taken from \cite{article9}.
     \label{b_heart_structure_new.png}}
\end{figure}

%%the primary function of the cardiovascular system is to supply body cells with nutrient material and carry away waste products
%%, and it does not care about poetry or emotions.

\subsection{The Blood Flow in the Caridovascular System}
For a greater understanding of how the heart functions, an e


%%which came first, the clot or the heart attack?
